\chapter{Einleitung}

\section{Hintergrund \& Motivation}

In Zeiten der Globalisierung ist es für Großkonzerne von großer Bedeutung, Prozesse zu vereinfachen, um Ressourcen und Kapazitäten einzusparen. Vor allem in der \ac{IT} ist Optimierungspotential vorhanden, da hier viele Prozesse und Arbeitsvorgänge immer noch nicht vereinheitlicht sind. Stattdessen gibt es noch innerhalb der verschiedenen Gesellschaften des ABB Konzerns unterschiedliche Arbeitsweisen, welche noch nicht standardisiert sind. Dies hat zur Folge, dass Prozesse unterschiedlich lang andauern und unterschiedlich viele Kosten verursachen. Auch werden Parallel die selben Aufgabenstellungen in verschiedenen Abteilungen auf die gleiche Art gelöst und könnten somit von einem standardisierten Prozess sehr profitieren. Beispielsweise werden beim Versand von Waren innerhalb der \ac{EU} \ac{LLE} ausgestellt, um den Zollbestimmungen gerecht zu werden. Bisher werden diese \ac{LLE} gesellschaftsabhängig erstellt, sodass bei Änderungen der Zollbestimmungen oder anderen Anpassungen, mehrere Dokument bearbeitet werden müssen. Hintergrund dieser Arbeit ist eine technische Zusammenführung selbiger Dokumente in eine homogene \ac{LLE} für alle Gesellschaften. Hiermit wird der Veränderungsprozess verkürzt und vereinfacht sowie sie Ausgabe von ABB-Dokumenten vereinheitlicht. Mit Hilfe der Adobe Interactive Forms im SAP sollen diese Anforderungen umgesetzt werden.

\section{Zielsetzung}

Ziel dieser Arbeit ist das Harmonisieren eines aktuell gesellschatsspeziefischen Formulars in Form einer Adobe PDF. Zwar sollen die verschiedenen Dokumente technisch in einem zusammengefasst werden, jedoch soll es weiterhin möglich sein, dynamisch Inhalte nur für bestimmte Bereiche darzustellen. Beispielsweise besteht die Anforderung, unterschiedliche Firmenlogos an verschiedenen Positionen anzuzeigen.

\section{Vorgehen}


Zunächst werden in Kapitel \ref{ch:Grundlagen} nötige Grundlagen für das Verständnis der Arbeit erläutert. Auf Basis der Ist-Analyse in Kapitel \ref{ch:Ist-Analyse} werden in Kapitel \ref{ch:Anforderungen} unternehmensspeziefische Anforderungen analysiert und ausgearbeitet. Die Konzeption des Entwurfes befindet sich im darauffolgenden Kapitel. Danach folgt die technische Umsetzung, welche auf der Anforderungsanalyse und dem Entwurf aufbaut. Abschließend werden die Erkenntnisse und Ergebnisse in einem Fazit widergespiegelt sowie einen Ausblick auf die Zukunft gemacht.