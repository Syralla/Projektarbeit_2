\chapter{Ist-Analyse}
\label{ch:Ist-Analyse}


	
	\section{Technischer Aufbau}
	\label{ch:ist-aufbau}
	23 Gesellschaften der ABB AG benutzen die \ac{LLE} für ihre Lieferungen. Viele von diesen Gesellschaften haben eine eigene Variation des Formulars. Die Dokumente unterscheiden sich meistens durch die Position des Briefkopfes oder im Firmen-Logo. 
	
	Aktuell wird das Formular mit "`Smart Forms"' umgesetzt. Smart Forms ist eine Formular-Technologie, entwickelt von SAP SE, welche seit Ende 1999 verfügbar ist. Damals sollte diese Variante der Dokumenterstellung den Bedarf an Programmierern für solche Formulare verringern. Mit Hilfe eines \ac{GUI} und anderen Werkzeugen, sollte das Erstellen und Verwenden von Dokumenten im SAP ohne eigene Programmierung ermöglicht werden.\footnote{Vgl. \cite{Hertleif.2003} S. 13}  
	
	In der Abbildung \ref{AufLLE} ist der inhaltliche Aufbau der \ac{LLE} dargestellt. 
	\begin{figure}[ht]
		\centering
		\makebox[\textwidth][c]{\includegraphics[width=1\textwidth]{img/LLE.png}}%		
		\caption{Aufbau der Lieferantenerklärung}
		\label{AufLLE}
		
	\end{figure}
	
	Das zweiseitige Anschreiben, der Zoll-Text und die Legende werden im Hochformat und die Materialliste im Querformat ausgegeben. Die Liste kann beliebig lang sein und wird auf neue Seiten mit demselben Layout umgebrochen. Alle Varianten der \ac{LLE} sind zu einem Formular zusammengetragen. Die variablen Inhalte, welche sich von Gesellschaft zu Gesellschaft unterscheiden, werden mit Hilfe von Anzeigebedingungen gesteuert. Die meisten Unterschiede befinden sich auf dem Anschreiben des Dokuments. Unterschiedliche Logos sowie verschiedene Positionen und Ausprägungen der Adressfelder sorgen für eine unübersichtliche Vorschau im "Form Painter" in Smart Forms. Der Form Painter zeigt eine Vorschau des Dokumenten-Layouts ohne dabei die Inhalte der verschiedenen Felder zu berücksichtigen.\footnote{Vgl. \cite{Hertleif.2003} S. 79-81}
	In Abbildung \ref{fig2} ist zu sehen, wie unübersichtlich dieses Layout im Fall der \ac{LLE} aussieht. Sobald Änderungen am Inhalt des Dokumentes von Nöten sind, müssen demnach alle Varianten des Formulars bedacht werden, um Fehler zu vermeiden. Dies führt zu erhöhtem Pflegeaufwand, sowie einem verlängertem Anpassungsprozess.
	
	\begin{figure}[ht]
		\centering
		\makebox[\textwidth][c]{\includegraphics[width=1\textwidth]{img/Smartform-Beispiel-1.png}}%		
		\caption{Lieferantenerklärung in Smart Forms}
		\label{fig2}
	\end{figure}
	
	Ein Ergebnis beim Zusammenführen der verschiedenen Varianten des Formulars in Smart Forms, ist auch, dass die verwendeten Texte an unterschiedlichen Stellen hinterlegt sind. Für den einen Teil der Texte werden Textbausteine verwendet, welche unabhängig von Formularen geändert werden können. Diese Bausteine werden im Formular aufgerufen. Der andere Teil der Texte ist jedoch direkt im Smart Form als Textfeld mit festem Inhalt angelegt. 
	
	Diese Inkonsistenz wird weiter in Form von kleinen \ac{ABAP}-Code Abschnitten in dem Formular geführt, welche die Formulardaten weiterverarbeiten, sowie kleinere Randinformationen zu Verfügung stellen sollen. Diese kleineren Funktionen sind auf das ganze Formular verteilt. Oftmals ist nicht erkennbar, wofür ein kleiner Codeabschnitt zuständig ist.
	
	Diese Punkte sorgen dafür, dass der Änderungsprozess zusätzlich verlängert wird, da bei einem so komplexen Dokument mehr Zeit eingeplant werden muss, da oftmals Mitarbeiter an dem Dokument Änderungen vornehmen müssen, ohne das Formular davor schon gekannt zu haben.\footnote{Durch den Einsatz von externen Dienstleistern ist dies oft der Fall} Der Einarbeitungsaufwand verlängert sich unnötiger weise durch solche Unstimmigkeiten.
	
	Wie in Abbildung \ref{fig3} zu sehen ist, wird die \ac{LLE} aktuell in drei Sprachen gleichzeitig erstellt. Obwohl es gesetzlich nicht so vorgegeben ist, stehen alle drei Sprachen gleichzeitig auf dem Dokument, um den zusätzlichen Aufwand einer Übersetzung zu vermeiden.
	
	\begin{figure}[ht]
		\centering
		\makebox[\textwidth][c]{\includegraphics[width=1\textwidth]{img/Sprachen.png}}%		
		\caption{Mehrsprachige Lieferantenerklärung}
		\label{fig3}
	\end{figure}
	


	Die Zuweisung eines Formulars zu einer Gesellschaft findet im Customizing statt. Das Customizing ist ein separater Bereich im SAP, in welchem Einstellungen, Funktionen und Prozesse gesteuert werden. Im Fall der \ac{LLE} wird hier festgelegt, für welche Verwaltungseinheit welches Formular verwendet wird. In Abbildung \ref{fig4} ist zu sehen, dass bei Smart Forms zusätzlich auch noch ein Formulartext festgelegt werden kann.
	
		
	\begin{figure}[ht]
		\centering
		\makebox[\textwidth][c]{\includegraphics[width=1\textwidth]{img/Customizing.png}}	
		\caption{Customizing der Lieferantenerklärung }
		\label{fig4}
	\end{figure}

	Grundsätzlich funktioniert die \ac{LLE} in ihrem jetzigen Zustand, aber weitere Änderungen seitens der Zoll-Vorgaben oder der Gesellschaften führen immer mehr zu zusätzlichem Aufwand\footnote{Siehe Ist-Analyse/Entwurf/durchführung.. Erwähnung welche Änderungen alleine in den letzten Monaten nötig waren}, welcher bei einer optimierten Lösung nicht nötig wäre. 

	\FloatBarrier
	
	\section{Schnittstelle in Smart Forms}
	
	Die Daten, welche zum Füllen der \ac{LLE} erforderlich sind, werden über eine Schnittstelle an das Formular übertragen. Diese Schnittstelle wird in Smart Forms definiert und besteht aus Import- und Export-Parametern, Tabellen und Ausnahmen. Die Import- beziehungsweise Export-Parameter beinhalten die Parameter, welche Daten dem Formular zur Verfügung stellen bzw. Daten von dem Formular aus zurück an das Druckprogramm zurückgeben\footnote{Vgl. \cite{Hertleif.2003} S. 169 - 173}. Im Falle der \ac{LLE} beinhaltet die Schnittstelle neben den Standard-Parametern zwei Standard-SAP-Strukturen bei den Import-Parametern. Diese beinhalten alle benötigten Daten, welche für die \ac{LLE} relevant sind. Diese Strukturen müssen ebenfalls in der Schnittstelle des neuen Formulars vorhanden sein. Mit Hilfe der Ausnahmen in der Schnittstelle werden Fehler definiert, welche beim Auftreten zu keinem Programmabbruch führen sollen. Im Zusammenhang zur \ac{LLE} sind keine weiteren Ausnahmen definiert.    
	\section{Globale Definitionen}
	
	Nicht alle benötigten Daten für das Formular können über Standard-SAP-Strukturen in der Schnittstelle bereitgestellt werden. Hierfür können Globale Definitionen erstellt werden, welche im gesamten Formular zur Verfügung stellen. Neben tatsächlichen Daten-Strukturen ist es, unter anderem, möglich Programmroutinen für die Initialisierung des Dokumentes einzufügen, sowie Formroutinen\footnote{Vgl. \cite{Hertleif.2003} S. 176}. Für die \ac{LLE} sind diverse Globale Daten definiert, welche im Formular an verschiedenen Stellen Anwendung finden. Ein großer Teil dieser Daten ist für das Auslesen von zusätzlichen Informationen aus dem SAP System da. Definitionen, welche im neuen Formular unvermeidlich sind, müssen übernommen werden.
	
	
	\section{Inhalte der \acs{LLE}}
	
	Die Elemente in Smart Forms werden in einer Ordnerstruktur angezeigt. Im Folgendem wird diese Struktur analysiert und in gesellschaftsspezifische, harmonisierbare und einheitliche Inhalte unterteilt.
	Im Anhang \ref{AN:Smart1} sind die Elemente des Anschreibens von ABB aufgelistet. Jeder Punkt, welcher mit einem blauen Symbol versehen ist, ist mit einer Bedingungen belegt, welche die Anzeige dieses Elements nur unter dieser Bedingung zulässt. Somit ist sofort ersichtlich, dass der Punkt "`ANSCHREIBEN\_ANFORDERN\_TEXT"' keine Bedingung hat, das heißt, dieses Element wird für jede Gesellschaft gleich benutzt. Jedes andere Element findet nur unter bestimmten Bedingungen Verwendung. Das gesamte Anschreiben ist unterteilt in folgende Bereiche:
	
	\begin{itemize}
		\item Anschreiben
		\item Logos
		\item Adresskopf für Empfänger
		\item Adresskopf für die absendende Gesellschaft
		\item Adresskopf für den absendenden Mitarbeiter
		\item Fußzeile
		\item Sonstiges
	\end{itemize}

	Die Bedingungen prüfen fast ausschließlich die \ac{AH}-Nummer ab. Diese Nummer stellt im Falle der \ac{LLE} die Unterteilung der verschiedenen Gesellschaften dar.
\FloatBarrier


	\subsection{Anschreiben}
	
	Das Anschreiben ist gleich für jede Gesellschaft jedoch ist es trotzdem mit dynamischen Inhalten gefüllt. Das Jahr, ab welchem die \ac{LLE} gültig ist, steht mehrfach im Text. Zusätzlich steht am Ende des Anschreibens der Firmenname sowie der Name des Ausstellers der \ac{LLE}. Diese Inhalte werden beim Drucken des Formulars eingesetzt.	
	
	\subsection{Logos}
	\label{ist_logos}
	
	Bei näherer Betrachtung der Logo-Elemente ist zu erkennen, dass die Grafiken in unterschiedlichen Weisen eingebunden werden. In Abbildung \ref{logo_smart} ist zu sehen, dass der Objektname, welcher auf die Grafik verweist\footnote{Grafiken können als "`GRAPHICS"' Objekt im SAP abgespeichert und über den definierten Namen an anderen Stellen eingefügt werden}, entweder fest eingetragen ist oder dynamisch eingefügt wird. 
	

	Es gibt insgesamt sechs Logo-Elemente, wobei zwei davon für eine Gesellschaft benutzt werden. Die Elemente mit den Endungen "`BJ"', "`23XX"', "`KAUFEL"' und "`AT"' sind nur für bestimmte Gesellschaften, während das reine "`LOGO"' Element für alle anderen Gesellschaften verwendet wird. Auffällig hierbei ist, dass für die \ac{AH} 3001, welche für die ABB Österreich steht, zwei Logo-Elemente benötigt werden.
	Über die Positionierung ist zu erkennen, das eines davon in der Fußzeile eingesetzt wird.
	
		\begin{figure}[ht]
		\centering
		\makebox[\textwidth][c]{\includegraphics[width=9cm]{img/logo_smartform.png}}	
		\caption{Unterschiedliche Versionen der Logo-Einbindung in Smart Forms}
		\label{logo_smart}
	\end{figure} 
	Da nicht einfach für jede Gesellschaft dasselbe Logo verwendet werden kann, auf Grund von unterschiedlichem Branding, können die Logo-Elemente nicht harmonisiert werden.
	
	\subsection{Adressköpfe}
	\label{ist:adr}
	
	Das Anschreiben der \ac{LLE} beinhaltet drei verschiedene Adressköpfe. Einen für die Empfängeradresse, einen für die Adresse der Gesellschaft, welche die \ac{LLE} erstellt hat, sowie einen Adresskopf für den Mitarbeiter dieser Gesellschaft. Je nach Gesellschaft unterscheiden sich die Adressfelder im Inhalt beziehungsweise in der Position im Formular. Die Position kann hierbei vereinheitlicht werden, da sie sich im Smart Forms Formular nur geringfügig unterscheidet und somit die Änderungen nicht sichtbar sein werden. 
	
	Teilweise können in Smart Forms Adressen als Textelemente dargestellt werden, welche über eine zugewiesene Adressnummer automatisch die zugehörigen Daten in einer definierbaren Struktur ausgeben. In Abbildung \ref{auto} ist die Definition eines solchen Feldes dargestellt. Die Adressnummer ist eine eindeutige Kennung für einen, im SAP System hinterlegten, Adressendatensatz.
			\begin{figure}[ht]
		\centering
		\makebox[\textwidth][c]{\includegraphics[width=9cm]{img/auto_adress.png}}	
		\caption{Unterschiedliche Versionen der Logo-Einbindung in Smart Forms}
		\label{auto}
	\end{figure} 
	
	 Der Adressblock für die Empfängerdaten beinhaltet einen automatisch erstellen Adresskopf und Zusatzdaten in Form von einem Ansprechpartner sowie der Lieferanten Nummer. Über dem Adresskopf des Empfängers befindet sich die Absenderadresse in einer kleineren Version. Dieses Element ist für alle Gesellschaften gleich, außer für die \ac{AH} 2000, welche für die Firma \ac{BJE} steht.Für diese Gesellschaft besteht dieses Feld an derselben Position. Jedoch beinhaltet das Element nicht nur die Absenderadresse, sondern zusätzlich noch die Empfängerdaten, welche für alle anderen Gesellschaften in dem, vorher erwähnten, separaten Element angelegt sind. Eine Möglichkeit der Harmonisierung ist somit gegeben, da es inhaltlich keine Unterschiede unter den Gesellschaften gibt.
	
	Die Kontaktdaten und Abteilungsdaten des Mitarbeiters, welcher die \ac{LLE} erstellt hat, werden auf dem Anschreiben oben rechts positioniert.
	Dem Smart Forms Formular kann entnommen werden, dass es aktuell vier Elemente für diesen Adressblock gibt, welche im Anhang \ref{AN:Smart1} unter den Namen "`WINDOW\_USER\_AND-
	\_DEPARTMENT"' erkennbar sind. Für diese Daten wird kein Adressenelement verwendet, stattdessen ist ein Text definiert, welcher mit Variablen versehen ist. Mit Hilfe dieser Variablen wird der Text mit den benötigten Adressdaten gefüllt. Da bei näherer Betrachtung der Elementinhalte kein Unterschied feststellbar ist, können auch diese Inhalte zusammengeführt werden.
	
	Die Adresse der ABB-Gesellschaft, welche die \ac{LLE} ausstellt befindet sich über dem Adressblock des Mitarbeiters. Drei Elemente werden im Moment für dieses Feld eingesetzt. Das Element mit den Namen "`SENDER\_ADDRESS23XX"' ist auf dem Anschreiben weiter oben platziert als die beiden anderen Elemente und ist per Bedingung nur für die \ac{AH}s 0050, 0051 und 0053 gültig. Inhaltlich variieren die Elemente nicht, da sie alle gleich definierte Adressenfelder beinhalten. Auffällig ist, dass für die \ac{AH} 2000 keine Absendeadresse vorhanden ist. Dieser Umstand ist darauf zurückzuführen, dass sich das Logo dieser Gesellschaft an der Position befindet, an welcher sich das Adressfeld bei den anderen Gesellschaften befindet. Die drei Elemente können dementsprechend zu einem zusammengeführt werden, da es keinen Grund gibt, das Adressfeld an unterschiedlichen Positionen anzuzeigen. Für die \ac{BJE}-Gesellschaft darf dieses Adressfeld nicht angezeigt werden.
	
	\subsection{Sonstige Inhalte des Anschreibens}
	\label{ist:rueck}
	
	Für die \ac{AH}-Nummern 0050, 0051 und 0053 existiert noch ein weiteres Element mit dem Namen "`RUECK Neues Fenster 1"'. Dieses Feld beinhaltet eine Erinnerung an den Empfänger, dass dieses Dokument nur an die Abteilung des Absenders zurück geschickt werden darf. Um den gleichen Inhalt im neuen Formular beizubehalten muss dieses Element auch im neuen Dokument vorhanden sein.
	
	Des Weiteren 
	
	\subsection{Zweite Anschreiben Seite}
	
	Im Anhang \ref{AN:Smart2} ist die Auflistung der Elemente auf der zweiten Seite des Anschreibens zu sehen. Es ist zu erkennen, dass alle Elemente der zweiten Anschreibenseite denen der ersten Seite gleichen. Der einzige Unterschied hierbei ist, der Zeitpunkt der Anzeige der einzelnen Elemente, welcher so eingestellt ist, dass diese nicht angezeigt werden. In Abbildung \ref{beding_smart} ist diese Bedingung dargestellt. Auf Grund dieser Einstellung ist die komplette zweite Seite nicht relevant für ein neues Formular und kann somit vernachlässigt werden.
	
	\begin{figure}[ht]
		\centering
		\makebox[\textwidth][c]{\includegraphics[width=9cm]{img/Anzeigebedingung_Smartform.png}}	
		\caption{Anzeigezeitpunkt der zweiten Seite in Smart Forms}
		\label{beding_smart}
	\end{figure}
	
	\subsection{Anschreiben der \acs{LE}}
	\label{ist:le}
	
	Zusätzlich zum Anschreiben der ABB gibt es noch ein Anschreiben, welches inhaltlich vom Zoll vorgegeben ist. Dieser Text beschreibt die Bedeutung des Dokumentes und bildet die tatsächliche "`Erklärung"' in der \ac{LLE}. Der Unterordner "`PREF Formulare 1"' im Anhang \ref{AN:Smart3} beinhaltet die Elemente dieser Seite. Zum einen beinhaltet der Ordner den vorgegebenen Zolltext und eine Logik für die Seitenzahlen und zum anderen ein Element, welches dynamische Inhalte des Textes füllt. Zu diesen dynamischen Inhalten gehören Angaben zum Datum, sowie Adress- bzw. Kontakt-Daten. Der Knoten "`DATEN ERMITTELN"' beinhaltet ein selbstgeschriebenes Programm, welches die Daten für den folgenden Teil der \ac{LLE} aufbereitet. Da die komplette Seite unabhängig von den Gesellschaften ist, bietet sich die Möglichkeit diese in das neue Formular zu übernehmen. 
	
	\subsection{Materialliste}
	
	Einen Teil der \ac{LLE} bildet eine Auflistung von Materialien. Diese Liste kann beliebig lang sein und besteht aus sieben Spalten. Im Anhang \ref{AN:Smart3} ist zu sehen, dass die Seite dieser Liste aus fünf Elementen besteht: einer Logik für die Seitenzahlen, Aussteller- und Empfänger-Daten der \ac{LLE}, Adress-Daten von Absender und Adressat und einem Element für die Liste selbst.
	In Abbildung \ref{list_smart} ist der Aufbau des Listen-Elementes dargestellt.
	
	\begin{figure}[ht]
		\centering
		\makebox[\textwidth][c]{\includegraphics[width=9cm]{img/Liste_Smartform.png}}	
		\caption{Aufbau des Listen-Elements in Smart Forms}
		\label{list_smart}
	\end{figure}
	
	Eine Schleife wird dafür benutzt, jede Zeile der Liste mit Inhalt zu füllen. Abschließend wird ein Kommandoelement für die Weiterleitung auf die nächste Seite benötigt, da beliebig viele Seiten mit dieser Liste gefüllt werden können.
	
	Die einzige Bedingung, mit welcher die Elemente dieser Seite verknüpft sind, ist dass ein präferenzberechtigtes Material\footnote{Siehe Kapitel \ref{LLE}} vorliegen muss. Da diese Bedingung für alle Gesellschaften gilt und sonst keine weiteren Bedingungen oder sonstige Unterschiede festzustellen sind, kann diese Seite ebenfalls in das neue Formular übernommen werden. 
	
	\subsection{Legende}
	
	Die Seite nach der Materialliste besteht aus einer Legende, welche Abkürzungen von Abkommen, die in der Materialliste verwendet werden, auflistet. Diese Liste ist immer identisch und kann somit direkt so im neuen Formular eingesetzt werden.
	
	
	\subsection{Sonstige Seiten}
	
	Im Anhang \ref{AN:Smart3} sind weitere Seiten aufgelistet, welche obsolete Inhalte vorhergegangener Versionen der \ac{LLE} sind. Diese Seiten werden aktuell durch Anzeigezeitpunkte ausgeblendet. Sie wurden bisher nicht gelöscht, um eventuelle Wiedereinführungen einfacher zu gestalten
	Das neue Formular wird nach den aktuellen Vorgaben\footnote{Vgl. \cite{ZOLL.2017}} des Zolls aufgebaut sein und somit diese Seiten nicht weiter beinhalten.
	
	
	\FloatBarrier
	\section{Anforderungen}
	\label{ch:Anf}
		Die Grundanforderung an das neue Formular ist, dass zum Ende der Arbeit inhaltlich dasselbe Dokument, auf Basis der Interactive Forms, ausgedruckt wird, wie mit Smart Forms. Optisch sollte kein Unterschied für die operativen Einheiten zu erkennen sein. Des Weiteren ist es sehr wichtig, dass die Fachbereiche nicht beeinträchtigt werden durch Einarbeitungszeit oder längere Umstellungsprozesse. 
	
		Gesellschaftsspezifische Inhalte, wie beispielsweise Logos, sollen weiterhin dynamisch angedruckt werden.
		Die Adressfelder, welche in dem Smart Forms Formular unterschiedlich sind, werden vereinheitlicht, um das Dokument technisch gesehen übersichtlicher zu gestalten. Trotz der Harmonisierung sollen weiterhin alle Daten angezeigt werden, die aktuell auf dem Formular stehen. 
		
		Um zukünftige Änderungen am Formular zu vereinfachen, müssen alle Elemente und Komponenten der Adobe \ac{PDF} kongruent und leicht verständlich benannt sein. Wird bei der Erstellung des Formulars kein System verfolgt, kann es dazu führen, dass erneut ein erhöhter Einarbeitungsaufwand erforderlich wird.
		
		Für die \ac{LE} ist es nicht von Nöten, eine optimierte Art der automatischen Übersetzung der Texte zu finden, da alle nötigen Sprachen (Deutsch, Englisch und Französisch) gleichzeitig im selben Formular aufgedruckt werden. Eine Übersetzung wäre jedoch durch die Standard SAP Funktion möglich.
		
