%
%   Prof. Dr. Julian Reichwald
%   auf Basis einer Vorlage von Prof. Dr. Jörg Baumgart
%   DHBW Mannheim
%
%
%	ACHTUNG: Für das Erstellen des Literaturverzeichnisses wird das modernere Paket biblatex
%			 in Kombination mit biber verwendet -- nicht mehr das ältere BibTex!
% 			 Bitte stellen Sie ggf. Ihre TeX-Umgebung
% 			 entsprechend ein (z.B. TeXStudio: Einstellungen --> Erzeugen --> Standard Bibliographieprogramm: biber)
%

\documentclass[
	12pt,
	BCOR=5mm,
	DIV=12,
	headinclude=on,
	footinclude=off,
	parskip=half,
	bibliography=totoc,
	listof=entryprefix,
	toc=listof,
	pointlessnumbers,
	plainfootsepline]{scrreprt}

%	Konfigurationsdatei einziehen
\input{config}

\begin{document}

%% BITTE GEBEN SIE HIER DEN TITEL UND DIE AUTORIN / DEN AUTOR DER ARBEIT AN!
%% DIESE INFORMATIONEN _MÜSSEN_ GESETZT SEIN, UM TITELBLATT, ABSTRACT UND 
%% EIGENSTÄNDIGKEITSERKLÄRUNG AUTOMATISCH ANZUPASSEN!
\TitelDerArbeit{Harmonisierung von gesellschaftsspezifischen Formularen mit Interactive Forms von Adobe am Beispiel der Lieferantenerklärung im SAP GTS}
\AutorDerArbeit{Simon Fischer}
\Firma{ABB}
\Kurs{WWI15AMC}

\begin{titlepage}
\begin{minipage}{\textwidth}
		\vspace{-2cm}
		\noindent \includegraphics[scale=0.71]{img/firmenlogo.jpg} \hfill   \includegraphics{img/logo.jpg}
\end{minipage}
\vspace{1em}
\sffamily
\begin{center}
	\textsf{\large{}Duale Hochschule Baden-W\"urttemberg\\[1.5mm] Mannheim}\\[2em]
	\textsf{\textbf{\Large{}Projektarbeit}}\\[3mm]
	\textsf{\textbf{\DerTitelDerArbeit}} \\[1.5cm]
	\textsf{\textbf{\Large{}Studiengang Wirtschaftsinformatik}\\[3mm] \textsf{Studienrichtung Application Management}}
	
	\vspace{3em}
	\textsf{\Large{Sperrvermerk}}
\vfill

\begin{minipage}{\textwidth}

\begin{tabbing}
	Wissenschaftlicher Betreuer: \hspace{0.85cm}\=\kill
	Verfasser/in: \> \DerAutorDerArbeit \\[1.5mm]
	Matrikelnummer: \> 2878271 \\[1.5mm]
	Firma: \> \DerNameDerFirma  \\[1.5mm]
	Abteilung: \> DE-IS \\[1.5mm]
	Kurs: \> \DieKursbezeichnung \\[1.5mm]
	Studiengangsleiter: \> Prof. Dr. Dennis Pfisterer  \\[1.5mm]
	Wissenschaftlicher Betreuer: \> Tarek Becker \\
	\> tarek@becker.ly \\
	\> +49 176 / 973 279 24 \\[1.5mm]
	Firmenbetreuer: \> Kevin Heid \\
	\> kevin.heid@de.abb.com \\
	\> +49 160 / 291 2790 \\[1.5mm]
	Bearbeitungszeitraum: \> 08.08.2017 -- 26.09.2017
\end{tabbing}
\end{minipage}

\end{center}

\end{titlepage}

\pagenumbering{Roman} % Römische Seitennummerierung
\normalfont
\hyphenpenalty=5000

%--------------------------------
% Verzeichnisse - nicht benötige Verzeichnisse bitte auskommentieren / löschen.
%--------------------------------

%   Sperrvermerk
\chapter*{Sperrvermerk}
Die vorliegende Arbeit enthält vertrauliche Daten der ABB AG. Sie darf als Leistungsnachweis des Studiengangs „Wirtschaftsinformatik“ an der Dualen Hochschule Baden-Württemberg Mannheim nur zu Prüfungszwecken zugänglich gemacht werden. 

Über den Inhalt ist Stillschweigen zu bewahren. Veröffentlichungen und Vervielfältigungen der Projektarbeit -auch auszugsweise- sind ohne ausdrückliche Genehmigung der ABB AG nicht gestattet.   
\cleardoublepage


%	Kurzfassung
\chapter*{Kurzfassung}
\begingroup
\begin{table}[h!]
\setlength\tabcolsep{0pt}
\begin{tabular}{p{3.7cm}p{11.7cm}}
Titel & \DerTitelDerArbeit \\
Verfasser/in: & \DerAutorDerArbeit \\
Kurs: & \DieKursbezeichnung \\
Ausbildungsstätte: & \DerNameDerFirma\\
\end{tabular}
\end{table}
\endgroup

In einem Großkonzern wie der ABB ist es von großer Wichtigkeit Prozesse zu standardisieren. Gerade im Bereich der IT ist dies wichtig, da die Effektivität durch vereinheitlichte Abläufe über verschiedene Organisationseinheiten mit gleichbleibender Qualität gesteigert wird, Reduktion von Kosten erfolgt, da Einzelfallbearbeitung entfällt und Lösungswege einheitlich genutzt werden können und die Standardisierung zukünftige Prozessverbesserungen durch die Messbarkeit ermöglicht. Im Rahmen dieser Arbeit soll ein Dokument erstellt werden, welches automatisch aus dem \ac{ERP} System "`SAP"' heraus erstellt wird. Dieses PDF soll sich dynamisch an den erstellenden Bereich der ABB anpassen. Es werden zunächst die Grundlagen der PDF Erstellung im SAP erläutert und die benutzte Technologie wird erklärt. Anschließend wird der Ist-Zustand untersucht und kritisch beurteilt. Zudem wird ein Soll-Konzept erstellt werden und eine Vorgehensweise geplant. Anhand dieses Plans wird anschließend die Umsetzung eines neuen Formulars konzipiert und aufgestellt. Abschließend wird das entstandene PDF mit dem Soll-Konzept abgeglichen und ein Fazit wird gezogen.




%	Inhaltsverzeichnis
\tableofcontents

%	Abbildungsverzeichnis
\listoffigures

%	Tabellenverzeichnis
%\listoftables

%	Listingsverzeichnis
% \lstlistoflistings
 
 

% 	Algorithmenverzeichnis
% \listofalgorithms

% 	Abkürzungsverzeichnis (siehe Datei acronyms.tex!)
\clearpage
\chapter*{Abkürzungsverzeichnis}	
\addcontentsline{toc}{chapter}{Abkürzungsverzeichnis}


\begin{acronym}[RDBMS]
	
	\acro{AA}{Asset Accounting}
	\acro{ABAP}{Advanced Buisiness Application Programming}
	\acro{ABB}{ASEA Brown, Boverie}
	\acro{AH}{Außenhandelsorganisation}
	\acro{ALCD}{Adobe LifeCycle Designer}
	\acro{ASEA}{Allmänna Svenska Elektriska Aktiebolaget}
	\acro{BBC}{Brown, Boveri \& Cie}
	\acro{BJE}{Busch-Jaeger Elektro}
	\acro{CO}{Controlling}
	\acro{CS}{Customer Service}
	\acro{DE-IS}{Information Systems Deutschland}
	\acro{DHBW}{Duale Hochschule Baden-Württemberg}
	\acro{ERP}{Enterprise Ressource Planning}
	\acro{EU}{Europ\"aischen Union}
	\acro{FI}{Finance}
	\acro{GTS}{Global Trade Services}
	\acro{GUI}{Graphical User Interface}
	\acro{HR}{Human Resources}
	\acro{IT}{Informationstechnik}
	\acro{LE}{Lieferantenerklärung}
	\acro{LLE}{Langzeit-Lieferantenerklärung}
	\acro{MM}{Material Management}
	\acro{PDF}{Portable Document Format}
	\acro{PM}{Plant Maintenance}
	\acro{PP}{Product Planning}
	\acro{QM}{Quality Management}
	\acro{SD}{Sales and Distribution}
	\acro{SQL}{Structured Query Language}
	
	
	
	
\end{acronym}




%--------------------------------
% Start des Textteils der Arbeit
%--------------------------------
\clearpage
\ihead{\chaptername~\thechapter} % Neue Header-Definition
\pagenumbering{arabic}  % Arabische Seitenzahlen
\textasciitilde

%	Anleitungs-Datei anleitung.tex einziehen. Auf diese Weise sollten Sie versuchen, für jedes einzelne
% Kapitel eine eigene Datei anzulegen und mittels input-Kommando einzuziehen.
\chapter{Einleitung}

\section{Hintergrund \& Motivation}

In Zeiten der Globalisierung ist es für Großkonzerne von großer Bedeutung, Prozesse zu vereinfachen, um Ressourcen und Kapazitäten einzusparen. Vor allem in der \ac{IT} ist Optimierungspotential vorhanden, da hier viele Prozesse und Arbeitsvorgänge immer noch nicht vereinheitlicht sind. Stattdessen gibt es noch innerhalb der verschiedenen Gesellschaften des ABB Konzerns unterschiedliche Arbeitsweisen, welche noch nicht standardisiert sind. Dies hat zur Folge, dass Prozesse unterschiedlich lang andauern und unterschiedlich viele Kosten verursachen. Auch werden Parallel die selben Aufgabenstellungen in verschiedenen Abteilungen auf die gleiche Art gelöst und könnten somit von einem standardisierten Prozess sehr profitieren. Beispielsweise werden beim Versand von Waren innerhalb der \ac{EU} \ac{LLE} ausgestellt, um den Zollbestimmungen gerecht zu werden. Bisher werden diese \ac{LLE} gesellschaftsabhängig erstellt, sodass bei Änderungen der Zollbestimmungen oder anderen Anpassungen, mehrere Dokument bearbeitet werden müssen. Hintergrund dieser Arbeit ist eine technische Zusammenführung selbiger Dokumente in eine homogene \ac{LLE} für alle Gesellschaften. Hiermit wird der Veränderungsprozess verkürzt und vereinfacht sowie sie Ausgabe von ABB-Dokumenten vereinheitlicht. Mit Hilfe der Adobe Interactive Forms im SAP sollen diese Anforderungen umgesetzt werden.

\section{Zielsetzung}

Ziel dieser Arbeit ist das Harmonisieren eines aktuell gesellschatsspeziefischen Formulars in Form einer Adobe PDF. Zwar sollen die verschiedenen Dokumente technisch in einem zusammengefasst werden, jedoch soll es weiterhin möglich sein, dynamisch Inhalte nur für bestimmte Bereiche darzustellen. Beispielsweise besteht die Anforderung, unterschiedliche Firmenlogos an verschiedenen Positionen anzuzeigen.

\section{Vorgehen}


Zunächst werden in Kapitel \ref{ch:Grundlagen} nötige Grundlagen für das Verständnis der Arbeit erläutert. Auf Basis der Ist-Analyse in Kapitel \ref{ch:Ist-Analyse} werden in Kapitel \ref{ch:Anforderungen} unternehmensspeziefische Anforderungen analysiert und ausgearbeitet. Die Konzeption des Entwurfes befindet sich im darauffolgenden Kapitel. Danach folgt die technische Umsetzung, welche auf der Anforderungsanalyse und dem Entwurf aufbaut. Abschließend werden die Erkenntnisse und Ergebnisse in einem Fazit widergespiegelt sowie einen Ausblick auf die Zukunft gemacht.
\newpage

\chapter{Grundlagen}
\label{ch:Grundlagen}



\section{SAP}
%SAP SE ist ein deutsches Software-Unternehmen mit Sitz in Walldorf. Als drittgrößter unabhängiger Softwareanbieter der Welt beschäftigt das Unternehmen weltweit mehr als 87 Tausend Mitarbeiter. SAP hat sich vor allem auf Unternehmenssoftware spezialisiert. Einen großen Teil der Firmenaktivitäten bildet die selbst entwickelte ERP Software SAP ERP.\footcite{SAPSE.2017}

ABB nutzt als \ac{ERP}-System die Standardsoftware SAP ECC \footnote{Im weiteren Text nur "`SAP"' genannt} der Firma SAP SE. Eine Standardsoftware ist ein Produkt, welches die allgemeinen Anforderungen des Nutzers - in diesem Fall ABB - erfüllt und bei speziellen Anforderungen individuell angepasst werden kann. Die Firma SAP SE bietet verschiedene modulare Lösungen an, welche einzeln genutzt werden können. Beispiel hierfür sind die \ac{SD}-, \ac{FI}-, \ac{MM}- und \ac{HR}-Module. 

SAP SE hat eigens für ihr \ac{ERP}-System die Programmiersprache \ac{ABAP} entwickelt. Grundsätzlich ist \ac{ABAP} eine Mischung aus \ac{SQL}- und Java-Befehlen. Dabei ist  die Programmiersprache so optimiert, dass effizient mit großen Datenbanken gearbeitet werden kann. In den letzten Jahrzehnten wurde \ac{ABAP} in sofern weiterentwickelt, dass  seit Release 4.6 im Jahre 1999 auch objektorientiertes Arbeiten möglich ist\footnote{Vgl. \cite{Keller.2001} S.16}. Das SAP-System  ist zum größten Teil mit \ac{ABAP} programmiert, sodass Anpassungen und eigene Programme auch in \ac{ABAP} geschrieben werden müssen. 

In dieser Arbeit wird ein Formular erstellt, welches im \ac{GTS}-Modul benutzt wird. Dieses Modul behandelt Prozesse bezüglich gesetzlicher Kontrollen bei In- und Export und deren Zollabwicklung, sowie die in dieser Arbeit behandelte Präferenzabwicklung.




\section{Interactive Forms by Adobe}

Seit 2002 arbeitet SAP SE mit der Software Firma Adobe zusammen, um eine neue Art der Dokumenten-Erstellung im SAP zu schaffen. Aus dieser Zusammenarbeit entstanden die "`Interactive Forms by Adobe"', welche seit Jahren den Standard für die PDF-Erstellung in SAP ERP bilden. Trotz dessen wird diese Technologie bei ABB erst seit einem Jahr eingesetzt.

Neue \ac{PDF}-Dokumente können größtenteils ohne Programmieraufwand erstellt beziehungsweise angepasst werden. Als Hauptwerkzeug dient hierbei der Adobe Lifecycle Designer, welcher mit Hilfe von grafischen Darstellungen einfacheres Arbeiten ermöglicht.\footcite{Bahr.2016}

Die Erstellung von Adobe \ac{PDF}s in SAP besteht aus drei Teilen:

\begin{itemize}
	\item Eine Schnittstelle, in welcher die Daten und Einstellungen der PDF festgelegt sind.
	\item Das eigentliche Formular, welches das Layout sowie den Inhalt des Dokumentes vorgibt.
	\item Ein Druckprogramm, welches die Schnittstelle mit dem Formular verbindet und die PDF druckt.
\end{itemize}

\subsection{Technischer Aufbau der Adobe PDF}
\label{ch:Schnittstelle}

Adobe Forms sind ähnlich unterteilt wie der Vorgänger Smart Forms. Es gibt eine Schnittstelle, in welcher die Datendefinition und die Formularlogik festgelegt werden. Neben den Import- und Export-Parameter für den Dokumentendruck sind auch die tatsächlichen Daten, die das Formular füllen hier definiert. Zusätzlich wird noch die Möglichkeit gegeben, die Daten mit Hilfe von \ac{ABAP}-Code bei der Initialisierung der Dokumentenerstellung anzupassen.

Basierend auf der Schnittstelle gibt es das eigentliche Formular. Jedes Formular muss einer Schnittstelle zugewiesen sein. Nur dann sind die in der Schnittstelle definierten Felder benutzbar. Hier ist zu beachten, dass mehrere Formulare dieselbe Schnittstelle nutzen können. \footnote{Vgl. \cite{Hauser.2015} S.125-128} 

Im Formular wird zuerst ausgewählt, welchen Anteil der Daten, der in der Schnittstelle hinterlegt ist, im Dokument verwendet wird und unter welchen Bedingungen diese ausgegeben werden. Erste Eigenschaften der Felder im Dokument müssen schon hier bestimmt werden wie beispielsweise, ob das Feld aktiv ist oder nicht.


Beim Aktivieren des Formulars wird im Hintergrund ein Funktionsbaustein erstellt, welcher die Schnittstelle zwischen der Datenbeschaffung und Ausgabe des Formulars darstellt. Ein Funktionsbaustein ist ein in sich selbst abgeschlossenes Teilprogramm, welches mit oder ohne Input-Parameter ausgeführt werden kann und mit Hilfe von Export-Parametern Werte zurück gibt. Dieser ist vor allem für Funktionen hilfreich, welche häufig in verschiedenen Programmen in SAP benötigt werden.

\subsection{Aufbau eines Adobe Formulars}
\label{ch:Aufbau}


Die Erstellung eines Dokumentes findet im sogenannten Form Builder statt. Hierbei handelt es sich um ein \ac{GUI} welches die Dokumenterstellung in zwei Teile spaltet - in Kontext und Layout. 

Im Kontext-Bereich stehen die Inhalte der konfigurierten Schnittstelle zur Auswahl. Aus dieser Auswahl wird eine Datenhierarchie festgelegt. Nur diese Inhalte stehen im  Layout zur Verfügung. Sollen bestimmte Bereiche des Dokumentes nur unter definierten Bedingungen angezeigt werden, können diese Einstellungen ebenfalls im Kontext-Bereich vorgenommen werden. Bei manchen Inhalten der \ac{PDF} gibt es Pflichtparameter welche ausgefüllt müssen, zusätzlich zu optional einzustellenden Eigenschaften.
 Elemente der \ac{PDF}, welche beispielsweise abhängig von der Ländersprache sind, müssen einen Wert für die zu verwendete Sprache beinhalten. In Abbildung \ref{figAD} ist dies für das Beispiel der Adressfelder zu sehen. Bei Adressfeldern ist die Ausgabe abhängig vom Absenderland.
 
 \begin{figure}[ht]
 	\centering
 	\makebox[\textwidth][c]{\includegraphics[width=1\textwidth]{img/Adressfeld.png}}%
 	
 	\caption{Eigenschaften eines Adressfeldes}
 	\label{figAD}
 	
 \end{figure}

Unabhängig von den Daten aus der Schnittstelle können zusätzliche Objekte wie Abbildungen und Textbausteine hinzugefügt werden. Der tatsächliche Inhalt dieser Objekte kann wiederum abhängig von den Schnittstellendaten sein. Somit bildet der Kontext-Bereich das Bindeglied zwischen den Formulardaten und dem Dokumentenlayout.\footnote{Vgl. \cite{Hauser.2015} S.145-146} 

Eingebettet in das SAP \ac{GUI} ist der \ac{ALCD}.a Mit Hilfe dieses Tools wird die Formularvorlage für den Druck einer Adobe PDF erstellt. Auf Basis der im Kontext definierten Daten wird somit das Layout des Formulars und der anzuzeigenden Inhalt festgelegt. Folgende Funktionen sind hierbei Schlüsselwerkzeuge bei der Erstellung eines Dokumentes: \footcite{Hauser.2015}

\begin{itemize}
	\item Hierarchie und Datenansicht: \\
		Diese zwei baumartigen Darstellungen stellen unterschiedliche Aspekte des Formulars dar. Die Hierarchie zeigt den strukturellen Aufbau des Dokuments, während die Datenansicht den Aufbau der Daten visualisiert.
	\item Bibliothek: \\
		Die Bibliothek bildet eine Sammlung von Feldtypen, welche standardmäßig für ein Dokument zur Verfügung stehen. Oft verwendete Felder, wie beispielsweise Bild- oder Text-Felder, können somit leichter genutzt werden.
	\item Objekt-Palette: \\
		In den verschiedenen Reitern der Objekt-Palette werden die Eigenschaften von Formularelementen festgelegt.
	\item Formulardesignfläche: \\
		Diese Fläche bildet den Hauptbereich des \ac{ALCD}, da hier die Konfiguration der Formularvorlage stattfindet. Unterteilt wird diese in die Designansicht, welche das Layout und den Inhalt der einzelnen Seiten definiert, sowie die Masterseiten, welche die Hintergründe und das Format der Designseiten vorgibt.
	\item Datenbindungen: \\
		Mit Hilfe der Objekt-Palette wird Feldern eine Datenbindung zugewiesen. Über diese Bindung wird definiert, welche Daten in dem jeweiligen Feld angezeigt werden.
	\item Felder im Fließtext: \\
		Dynamische Inhalte, welche im Fließtext vorkommen, werden mit dieser Funktion eingefügt. Diese Felder werden beim Drucken der PDF gefüllt und der Fließtext wird um das Feld herum dementsprechend angepasst.
	\item Tabellen: \\
		Da die Größe von Tabellen meistens erst zum Zeitpunkt des Druckes feststeht, muss eine dynamische Ausgabe ermöglicht werden. Mit dieser Funktion können Tabellen erstellt und die Ausgaberegeln festgelegt werden.
	\item Seitenumbrüche: \\
		Bei Dokumenten mit dynamischen Inhalten ist es mit Hilfe dieser Funktion möglich, bedingte Seitenumbrüche festzulegen, um gegebenenfalls Inhalte auf einer beliebigen Anzahl neuer Seiten mit einem festgelegten Layout darzustellen.
\end{itemize} 
       
Adobe Forms können zusätzlich dazu benutzt werden, interaktive Formulare zu erstellen. Diese \ac{PDF}s können Elemente enthalten, welche nach dem Druck ausgefüllt werden können. Hierbei ist es möglich Auswahlfelder zu erstellen beziehungsweise den Aufbau des Dokuments dynamisch an die Eingaben anzupassen. Diese interaktive Funktionen werden in dieser Arbeit nicht weiter ausgeführt.
\FloatBarrier
\section{Langzeit-Lieferantenerklärung}
\label{LLE}

 Lieferantenerklärungen sind Dokumente, welche bei einer Lieferung Auskunft bezüglich des Herkunftslandes der gelieferten Materialien gibt. Diese Herkunftsangaben sind Voraussetzung für die Inanspruchnahme einer Zollpräferenz. Diese Zollpräferenz reduziert den Zollsatz für ein präferenzberechtigtes Material, in der Regel, auf null Prozent.\footnote{Vgl. \cite{Schnellenbach.2015} S. 351} Grundsätzlich findet die \ac{LE} bei Warenbewegungen innerhalb der \ac{EU} Anwendung. Die \ac{LLE}, welche als Beispiel-Dokument für diese Arbeit verwendet wird, ist eine einmalige Erklärung, welche für gleiche Lieferungen über einen maximalen Zeitraum von zwei Jahren gilt.\footnote{Vgl. \cite{ZOLL.2017}} 2016 erstellte ABB alleine in Deutschland dieses Dokument 1800 mal.\footnote{Diese Anzahl war das Ergebnis eines intern entwickelten Reports, welcher die ausgedruckten Dokumente im SAP gezählt hat} Text und Aufbau der Lieferantenerklärung entsprechen gesetzlichen Vorgaben. Dementsprechend wird in der folgenden Arbeit nur auf die technische Umsetzung eingegangen.

\newpage
\chapter{Ist-Analyse}
\label{ch:Ist-Analyse}

\begin{itemize}
	
	\item Verschiedene Dokumente für verschiedene Bereiche
	
	Knapp 40 Gesellschaften der ABB AG benutzen die \ac{LLE} für ihre Lieferungen. Viele von diesen 40 haben ihre eigene Variation des Formulars. Die Dokumente unterscheiden sich meist nur durch die Position des Briefkopfes oder im Firmen-Logo. 
	
	Aktuell wird das Formular mit "`Smart Forms"' umgesetzt. Smart Forms ist eine Formular-Technologie, entwickelt von SAP für SAP, welche seit Ende 1999 verfügbar ist. Damals sollte diese Variante der Dokumenterstellung den Bedarf von Programmierexperten für solche Formulare verringern. Mit Hilfe eines \ac{GUI} und anderen, Programmierlosen, Tools sollte das Erstellen und Benutzen von Dokumenten im SAP erleichtert werden.  
	
	
	Unübersichtlichkeit bei der Wartung
	
	Übervolles Formular
	
	Smartform
	
	Problemstellung
	
	Anforderungen
		- Pflegbarkeit
		- Vereinheitlichung wo möglich
		- Übersetzung
		- bedingte Ausgabe von bestimmten Elementen
		- optisch so wenig unterschied wie möglich zu davor
	
	
	\item Smartform
	\item Mehrfachnennung von Feldern \\
	
	\includegraphics[width=0.65\paperwidth, height=0.4\paperheight]{img/Smartform-Beispiel-1.png}
	\item Unübersichtlichkeit / nicht Wartbar 
\end{itemize}
\newpage
\chapter{Entwurf}

Nachdem in dem Vorhergegangenen Kapitel der aktuelle Stand des Formulars erläutert, sowie die Anforderungen zusammengetragen wurden, werden im folgendem Kapitel selbige Anforderungen in Kriterien umgewandelt und bewertet. Anhand dieser Kriterien werden mögliche Lösungsansätze verglichen. Somit wird ein optimaler Lösungsweg erarbeitet und in einen konkreten Plan umgewandelt.

\section{Wahl der Technologie}
\label{ch:Techn}

Die Verwendung von Adobe Interactive Forms ist eine ABB interne Vorgabe für diese Arbeit. Bis vor kurzem nutzte ABB bzw. die DE-IS nur Smart Forms im SAP.\footnote{Siehe Kapitel \ref{ch:Grundlagen}} Die Modernisierung bzw. Digitalisierung von Geschäftsprozessen führt jedoch dazu, dass neue Anforderungen an die automatische Dokumentenerstellung gestellt werden.\footnote{Siehe Kapitel \ref{ch:Ausblick}} Anforderungen welche die Smartforms in naher Zukunft nicht mehr erfüllen werden können, da neue Entwicklungen seitens SAP SE hauptsächlich für die neuere Technologie, die Interactive Forms, entwickelt werden.

Ein weiterer Grund für den Technologien Wechsel ist die Übersichtlichkeit der Inhalte. Die Aufspaltung in Schnittstelle und Formular bei den Interactive Forms führt dazu, dass das Layout getrennt von den Formularinhalten betrachtet bzw. angepasst werden kann. 


\section{Bewertung der Lösungsansätze}
\label{ch:Bewertung}

Oftmals sind mehrere Lösungsansätze für spezifische Probleme möglich. Um diese miteinander zu vergleichen wurde in Abbildung \ref{figNutz} eine Gewichtung, der in Kapitel \ref{ch:Anf} beschriebenen Anforderungen, durchgeführt\footnote{Diese Gewichtung fand in Zusammenarbeit mit dem  Leiter des GTS-Modul  der DE-IS statt}. 

\begin{figure}[ht]
	\centering
	\makebox[\textwidth][c]{\includegraphics[width=1\textwidth]{img/Nutzwert.png}}%		
	\caption{Nutzwertanalyse der Anforderungen}
	\label{figNutz}
\end{figure}

Diese Gewichtung kann benutzt werden um zwischen Lösungsalternativen wählen zu können. Es ist erkennbar, dass vor allem inhaltliche und optische Unterschiede vermieden werden müssen. Es muss ein Kompromiss gefunden werden zwischen der technischen Übersichtlichkeit und der Erfüllung der Anforderungen des neuen Formulars.


\section{Automatische Migration}
\label{ch:Migration}

Marcel Schmiechen beschreibt in seinem Buch über Interaktive Formulare im SAP \footcite{Schmiechen.2016} in Kapitel 8.4 das automatische Migrationsverfahren im SAP-Standard im Vergleich zu einer Neuentwicklung als nicht lohnenswert.\footnote{Vgl. \cite{Schmiechen.2016} S.189} Gerade bei größeren Dokumenten kommt es zu Fehlern bei der Migration. Dies hat auch der Test an Hand der \ac{LLE} bestätigt. Die Schnittstelle konnte fehlerfrei übernommen werden, jedoch führte die Migration zu großen Fehlern in dem Formular Teil. In Abbildung \ref{Migration} ist zu sehen, dass teilweise Attribute von Elementen nicht befüllt werden konnten. In diesem Fall fehlte vor allem das Absenderland für die verwendeten Addressknoten. Zusätzlich führten lange Namen von Elementen der Smart Form zu weiteren Fehlern, da die maximale Namenslänge bei den Interactive Forms zulange Namen abschneidet. Das führt dazu, dass Elemente gleich benannt sind.
\begin{figure}[ht]
	\centering
	\makebox[\textwidth][c]{\includegraphics[width=1\textwidth]{img/Migration_Test.png}}%		
	\caption{Fehler bei der Migration der \acs{LLE}}
	\label{Migration}
\end{figure}
Um das Layout zu überprüfen, musste das Formular zunächst überarbeitet werden, sodass die Fehlermeldungen nicht mehr die Aktivierung des Dokumentes verhindern. Nachdem die nötigen Änderungen durchgeführt wurden, konnte das Layout überprüft werden. 

Zu Beginn wurde der Aufbau des Dokumentes überprüft. Für jede nötige Seite der \ac{LLE} war zwar eine Masterseite vorhanden, jedoch nur für zwei Seiten auch eine Designseite. Das hat zur Folge, dass bei der Verwendung dieses Formulars nur diese beiden Seiten ausgedruckt werden würden. Unabhängig vom Aufbau und Inhalt des migrierten Formulars konnten noch weitere technische Fehler gefunden werden. Ein schwerwiegendes Problem ist, dass jedes Feld mit dynamischem Inhalt nur über den Namen mit dem zugehörigen Datensatz verbunden ist. Diese Zuordnung funktioniert zwar, jedoch ist sie sehr fehleranfällig bei Namensänderungen beziehungsweise neuen Zuordnungen, da so an mehreren Stellen die Änderungen vorgenommen werden müssten. In Abbildung \ref{Namen-Fehler} ist zu sehen, dass SAP diese Elemente mit einer Warnung versehen hat.

\begin{figure}[ht]
	\centering
	\makebox[\textwidth][c]{\includegraphics[width=1\textwidth]{img/Namen-Fehler.png}}%		
	\caption{Warnung bei der Verwendung von "Namen verwenden"}
	\label{Namen-Fehler}
\end{figure}

Neben der Struktur und den Datenbindungen kam es auch bei anderen Funktionen zu Fehlern bei der Migration. Die Tabelle konnte nicht in einer verwendbaren Weise übertragen werden und Texte, welche über Bedingungen in mehreren Sprachen vorhanden waren, wurden untereinander angezeigt. Da der Aufwand für die Überarbeitung dieser ganzen Fehler sehr hoch ist, vor allem wenn eine Harmonisierung beziehungsweise Optimierung der Inhalte stattfinden soll, wird von dieser Weise der Migration des Formularteils abgesehen. Die migrierte Schnittstelle kann jedoch verwendet werden. 

\section{Schnittstelle}

In der Schnittstelle werden die Inhalte des Formulars, welche aus dem SAP System stammen, definiert, sowie weitere benötigte Strukturen festgelegt.\footnote{Siehe Kapitel \ref{ch:Schnittstelle}}
Die tatsächlichen Inhalte der \ac{LLE} bleiben gleich, egal welche Technologie benutzt wird. Somit kann die Schnittstelle aus der Formularschnittstelle der Smartform-\ac{LLE} abgleitet werden. Für die Erstellung der Schnittstelle kann daher die SAP Standard Migration benutzt werden.\footnote{Siehe Kapitel \ref{ch:Migration}}


\section{Formular}



Das Formular wird auf Basis der migrierten Schnittstelle erstellt. 

\subsection{Allgemein}

Für das Anschreiben der ABB muss eine Seite im Layout-Bereich des Formulars erstellt werden. Die Inhalte des Anschreibens sind in den folgenden Kapiteln erläutert. Allgemein muss die Schriftart und Schriftgröße aus dem Smart Forms Formular übernommen werden. Bezüglich der Benennung der verschiedenen Objekte und Elemente gilt es sprechende, erklärende und eindeutige Namen zu verwenden. Um die dynamischen Inhalte in den statischen Text des Anschreiben einzufügen werden Fließtextfelder\footnote{Siehe Kapitel \ref{ch:Aufbau}} benutzt\footnote{Vgl. \cite{Hauser.2015} S. 319}.

\subsection{Anschreiben - Logos}

In Kapitel \ref{ist_logos} wurde festgestellt, dass Logos weiterhin gesellschaftsspezifisch verwendet werden müssen. Dementsprechend muss für jedes der sechs Logos ein eigenes Grafik-Element im Kontext-Bereich angelegt sowie im Layout-Bereich positioniert werden. Für die dynamische Anzeige der Logos werden diese Elemente mit einer Bedingung, welche die \ac{AH}-Nummer abprüft, versehen. Die tatsächlichen Logos werden, wie auch in Smart Forms, über den Namen des "`GRAPHICS"' Objektes eingebunden. Im Layout-Bereich werden die Logos als Bildfelder eingefügt und, kongruent zum alten Formular, positioniert. 

\subsection{Anschreiben - Adressköpfe}

Eine automatische Darstellung von Adressen ist im gleichem Sinne bei Interactive Forms möglich. Hierzu wird ebenfalls eine Adressnummer benötigt und zusätzlich das Land des Absenders\footnote{Siehe Kapitel \ref{Migration}}.

Die Position der Adressfelder ist in dem Smart Forms Formular nicht standardisiert und nicht allgemein für alle Gesellschaften gleich. Für das neue Formular soll die Position gleich für alle Gesellschaft sein um möglichst wenige verschiedene Elemente im Formular zu haben. Für die Festlegung der Ausrichtung der Empfänger Adresse wird ein Standard Briefumschlag mit Fenster als Referenz genommen. Somit wird der Versand der \ac{LLE} per Post vereinfacht ohne das jetzige Layout des Dokumentes zu beeinträchtigen. Der Adressblock des Empfängers sowie die schmale Variante des Absenders kann in einem Element zusammengefasst werden. Um die unterschiedliche Formatierung innerhalb des Elementes nutzen zu können wird von einem automatischen Adressblock abgesehen. Stattdessen wird die Adresse mit Hilfe von dynamischen Felder im Fließtext dargestellt. Der Vorteil dieser Variante ist, dass nur ein Element für alle Gesellschaften verwendet werden kann aber trotzdem variable Inhalte möglich sind.
 Würde stattdessen ein automatisches Adressfeld benutzt werden müssten mehrere Elemente angelegt und positioniert werden um verschiedene Formatierungen der Inhalte beizubehalten.
 
Für die Adressdaten der Absender Gesellschaft, auf der rechten Seite des Anschreibens, kann ein automatisch erstelltes Adresselement benutzt werden. Hierfür muss lediglich ein Element im Layout-Bereich hinzugefügt werden welches auf das Adresselement im Kontextbereich referenziert. Die Definition dieses Elements kann hierbei aus dem Smart Forms Formular übernommen werden. Über eine Bedingung wird dieses Feld anschließend für die \ac{AH} 2000 \footnote{Siehe Kapitel \ref{ist:adr}} ausgeblendet.

Die Kontaktdaten des Mitarbeiters werden ebenfalls mit Hilfe von Feldern im Fließtext dargestellt, da nicht alle benötigten Informationen Adressdaten sind, welche über einen Adressenbaustein eingefügt werden könnten. Hierfür wird ein weiteres Element benötigt, womit die Adressen mit drei Elementen vollständig wären. Dieses letzte Elemente wird inhaltlich und bedingungslos für alle Gesellschaften verwendet.

\subsection{Anschreiben - Sonstige Inhalte}

Das in Kapitel \ref{ist:rueck} erläuterte Info-Feld muss ebenfalls in das neue Formular übertragen werden. Für die Umsetzung gibt es zwei Möglichkeiten:

\subsubsection{Text als Feld im Fließtext}

Indem der anzuzeigende Text in einer Variable in der Schnittstelle eingefügt wird, kann der Text über ein Feld im Fließtext angezeigt werden. Hierfür ist es erforderlich, dass sowohl die Schnittstelle also auch das Formular angepasst werden, da der Text zunächst erst in einer Variable abgespeichert werden muss. Sollten Änderungen am Text nötig sein müssten somit an mehreren Stellen Anpassungen vorgenommen werden. Dies steht nicht im Sinne der Optimierung des Formulars.

\subsubsection{Textbaustein}

Der Inhalt wird als Textbaustein angelegt, welcher für die relevanten Gesellschaften eingelesen werden kann. Die bedingte Anzeige ist somit ersichtlich und der Text an einer genauen Stelle verfügbar für eventuell nötige Anpassungen. 

\subsection{Anschreiben der \acs{LLE}}

Für das Anschreiben der \ac{LLE} muss zunächst eine Seite erstellt werden welche in die Seitenzählung miteinbezogen wird. Der Text des Anschreibens kann aus dem Smart Forms Formular übernommen werden\footnote{Siehe Kapitel \ref{ist:le}}. Hierfür wird ein Textfeld benötigt welche mit Hilfe von Felder im Fließtext mit dynamischen Inhalten, beispielsweise Angaben zum Gültigkeitszeitraum, erweitert werden. 

\subsection{Materialliste}

Für Seiten der Materialliste sind mehrere Schritte zu beachten. Zunächst gilt es die Seite, abgesehen von der tatsächlichen Liste, zu erstellen. Für die Adressdaten sowie die Verwaltungseinheit werden Felder im Fließtext verwendet. Diese Kopfdaten werden über der Liste angeordnet, konvergent zum Smart Forms Formular. Anschließend muss die Materialliste erstellt werden. Hierfür muss zunächst, ähnlich wie bei Smart Forms, eine Schleife im Kontextbereich des Formulars definiert werden. Über diese Schleife ist es möglich die einzelnen Zeilen der Materialliste im Formular anzuzeigen. Im Layoutbereich wird anschließend die Liste erstellt mit einer Kopfzeile, welche die Spaltenbezeichnungen beinhaltet. Abschließend muss die Seite der Matierialliste in einer Weise konfiguriert werden, dass bei einer längeren Liste die Tabelle auf die nächste Seite umgebrochen wird. Hiebei darf kein Umbruch innerhalb einer Zeile möglich sein. Des Weiteren muss jede benötigte Folgeseite ebenfalls die Kopfdaten sowie die Kopfzeile der Materialliste darstellen. Ein Element für die Anzeige der Seitenanzahl wird am unteren Ender der Seite eingefügt. Nach der letzten Listenseite muss die anschließende Formularseite eingefügt werden.

\subsection{Legende}
Die letzte Seite des Formulars beinhaltet die Legende\footnote{Siehe Kapitel \ref{ist:leg}}.
Die Legende ist ein statischer Text und ist für alle Gesellschaften gleich. Daher wird der Text als Textbaustein in das Formular eingebunden, da so Änderungen zentral vorgenommen werden können. Eine Alternative zu dem Textbaustein wäre den Text direkt in das Formular in ein Textfeld zu schreiben. Dies würde dazu führen, dass für jede Änderung das Formular bearbeitet werden muss. Der Textbaustein dagegen, kann sehr schnell angepasst werden.



\section{Customizing}

Nachdem das Formular erstellt wurde wird es im Customizing jeder Gesellschaft zugeordnet\footnote{Siehe Kapitel \ref{ch:ist-aufbau}}. Somit ist die technische Umstellung auf das neue Formular abgeschlossen.



\section{Dokumentation}

Um zukünftige Bearbeitung der PDF zu vereinfachen wird eine Dokumentation nötig sein. Hier wird der fachliche Hintergrund des Dokumentes erläutert sowie die technische Umsetzung dokumentiert. Erklärungen von komplizierten Inhalten sowie generelle Informationen zur Umsetzung der \ac{LLE} werden ebenfalls in der Dokumentation festgehalten. 

Es ist nur sinnvoll die Dokumentation direkt an das Formular anzufügen in SAP. In der Transaktion SFP kann zu einem Formular eine Dokumentation hinterlegt werden. Diese ist standardmäßig unterteilt in folgende Unterpunkte:

\begin{itemize}
	\item Verwendung
	\item Einschränkungen
	\item Aufruf
	\item Kontext
	\item Layout
	\item Weitere Hinweise
	
\end{itemize}

Mit Hilfe dieser Unterteilung wird das neue Formular beschrieben und erläutert.
\newpage

\chapter{Umsetzung}

 Nachdem in dem Vorhergegangenen Kapitel der aktuelle Stand des Formulars erläutert, sowie die Anforderungen zusammengetragen wurden, wird in dem folgendem Kapitel ein Entwurf erstellt, welcher diese Anforderungen erfüllt. Anschließend wird die Umsetzung der Umstellung auf Adobe PDFs beschrieben.

 \section{Technischer Entwurf}
 

 
 \begin{itemize}
 	\item Schnittstelle
 	\item Formular
 	\item Druckprogramm
 	\item Migration von Smart Form -> Migration \footnote{Zitat \cite{Schmiechen.2016} S.189}
 \end{itemize}

\section{Dokument Erstellung}
\subsection{Schnittstelle}
	\begin{itemize}
		\item Import
		\item Globale Daten
		\item Initialisierung -> Code Initialisierung -> Beschaffung von non-Sap Standard Daten da kein Druckprogramm. -> extra Logo Beschaffung AT -> Valid From Year Berechnung
	\end{itemize}
\subsection{Masterseiten}
  \begin{itemize}
  	\item Hoch/Querformat
  	\item Follow-Up Seiten bei Tabellen -> Reihenfolge der Seiten und Nummerierung 
  \end{itemize}

\subsection{Design Seiten}

	\begin{itemize}
		\item Bindung zu Masterseiten
	\end{itemize}
\section{Einzelne Felder}
	\begin{itemize}
		\item Textfelder -> Position
		\item Bindung von Inhalten -> Beispiel Datum
		\item dynamische Felder -> Fließfelder -> Datum und co
	\end{itemize}
\section{Address Kopf}

	\begin{itemize}
		\item Inhalt über automatische Adressfeld? -> oder doch einzelne Werte?
		\item Position -> allgemein außer oben rechts bei BJ weil da das Logo ist
	\end{itemize}
\section{Dynamische Anzeige}

	\begin{itemize}
		\item Logoanzeige
		\item Bedingte Anzeige über Kontext im Layout -> IF AH = Bla dann Bla
		\item Logo Abspeicherung im System
		
		\item Tabellen anzeige -> Wie funktionieren Tabellen -> mehrseitige Ausgabe bei viel Inhalt
		
	\end{itemize}
\section{Übersetzung}

	\begin{itemize}
		\item Übersetzung mit Standard Variante -> nicht gut weil die Zeile/Zeile Funktion nicht funktioniert
		\item alternative -> Texte in Steuertabelle -> Nachteile: keine Fließfelder möglich bzw. nur schwer, Layout ist nicht sichergestellt bei Textänderungen
		\item andere alternative -> doppelte/dreifache Felder in verschiedenen Sprachen -> Nachteile: Absolutes Chaos, schlechte Wartbarkeit
	\end{itemize}

\section{Ausgabe}

	\begin{itemize}
		\item Beispiel eines Drucks -> dynamische Ausgabe zeigen -> zwei verschiedene Anschreiben in den Anhang?
	\end{itemize}


\newpage

\chapter{Ausblick und Fazit}

Unabhängig von dem Formular, welches Thema dieser Arbeit ist, gibt es noch viele weitere Dokumente im DAS System die noch mit Hilfe von Smart Forms umgesetzt sind. Auf Basis dieser Arbeit können Stück für Stück diese Dokumente mit Interactive Forms von Adobe ersetzt werden. 
\newpage

%	Literaturverzeichnis
\ihead{} % Neue Header-Definition
\pagenumbering{Roman}
\setcounter{page}{7}
\printbibliography[title=Quellenverzeichnis]
\cleardoublepage


\newgeometry{top = 10mm}

% Der Anhang beginnt hier - jedes Kapitel wird alphabetisch aufgezählt. (Anhang A, B usw.)
\appendix
\ihead{\appendixname~\thechapter} % Neue Header-Definition

% appendix.tex einziehen
\chapter{Anhang}
\section{Abbildungen}

\begin{figure}[ht]
	\centering
	\makebox[\textwidth][c]{\includegraphics[width=1\textwidth]{img/Smartform_Struktur_1.png}}%		
	\caption{Struktur der \acs{LLE} in Smart Forms}
	\label{AN:Smart1}
\end{figure}

\restoregeometry

\begin{figure}[ht]
	\centering
	\makebox[\textwidth][c]{\includegraphics[width=1\textwidth]{img/Smartform_Struktur_2.png}}%		
	\caption{Struktur der \acs{LLE} in Smart Forms}
	\label{AN:Smart2}
\end{figure}

\begin{figure}[ht]
	\centering
	\makebox[\textwidth][c]{\includegraphics[width=1\textwidth]{img/Smartform_Struktur_3.png}}%		
	\caption{Struktur der \acs{LLE} in Smart Forms}
	\label{AN:Smart3}
\end{figure}

\FloatBarrier

\section {Beispiel Dokumente}

\begin{figure}[ht] \centering{
		\includegraphics[scale=0.51]{at.pdf}}
	\caption{Anschreiben der \acs{LLE} per Smart Forms für die \acs{AH} 3001}
		\label{AN:lleat}
\end{figure} 

\begin{figure}[ht] \centering{
		\includegraphics[scale=0.70]{bje.pdf}}
	\caption{Anschreiben der \acs{LLE} per Smart Forms für die \acs{AH} 2000}
	\label{AN:llebj}
\end{figure} 






% Ehrenwörtliche Erklärung ewerkl.tex einziehen
\input{ewerkl.tex}


\end{document}
