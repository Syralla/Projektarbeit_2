\chapter{Ausblick und Fazit}
\label{ch:Ausblick}

Das Hauptziel der vorliegenden Arbeit war es, mehrere, nicht standardisierte Formulare aus einer älteren Technologie, Smart Forms,  in ein neues Adobe Interactive Forms Formular zusammen zuführen. Dazu wurde zunächst ein nötiges Vorwissen für diese Arbeit in den Grundlagen geschaffen um anschließend den Ist-Zustand zu analysieren. Aus dieser Analyse wurden die kritischen Punkte herausgezogen und in Anforderungen umformuliert. Diese Anforderungen wurden anschließend mit Hilfe eines Entwurfs umgesetzt. Anhand dieses Entwurfs wurde die Umstellung auf die neue Formulartechnologie erläutert.

Wichtige Anforderungen waren eine Vereinheitlichung der verschiedenen Varianten der Lieferantenerklärung, ohne Verluste seitens des Inhaltes. Mit Hilfe einer Analyse der Elemente des Smart Forms-Formulars, konnten redundante Inhalte identifiziert und zusammengeführt beziehungsweise, wenn nötig, komplett entfernt werden. Gesellschaftsspezifische Inhalte wurden, mit Hilfe von Anzeigebedingungen, beibehalten.

Unabhängig vom Inhalt des Formulars, sollte die technische Umsetzung ebenfalls einheitlich und übersichtlich durchgeführt werden. Durch eine einheitliche Benennung der Elemente der \ac{PDF}, wurde der Inhalt und die Funktionsweise verdeutlicht. Des Weiteren wurde darauf geachtet, dass gleichen Funktionen in einer festgelegten Weise umgesetzt wurden. Beispielsweise wurden alle statische Texte in Textbausteine ausgelagert. Des Weiteren wurden inaktive Elemente des alten Formulars nicht in das neue übernommen um so die Übernahme von Altlasten, welche eventuell nie wieder benutzt werden, zu vermeiden. Im Anhang ist ein Beispiel der \ac{LLE} mit Interactive Forms dargestellt.

Zusammenfassend ist die Harmonisierung des Dokumentes gelungen. Durch die vielseitigen Funktionen der Adobe \ac{PDF} konnte ein Standard gebildet werden, welcher den Prozess der Lieferantenerklärung vereinheitlicht. Unabhängig von diesem Formular, gibt es noch viele weitere Dokumente im SAP System von ABB, welche immer noch mit Smart Forms auf verschiedene Weise umgesetzt sind. Auf Basis dieser Arbeit können Stück für Stück diese Dokumente mit den \ac{PDF}s von Adobe vereinheitlicht werden.

 
