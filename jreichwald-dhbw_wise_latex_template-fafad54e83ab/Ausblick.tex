\chapter{Ausblick und Fazit}
\label{ch:Ausblick}

Das Hauptziel der vorliegenden Arbeit war es, mehrere, nicht standardisierte Formulare aus einer älteren Technologie - Smart Forms - in ein neues Adobe Interactive Forms-Formular zusammenzuführen. Dazu wurde zunächst ein für diese Arbeit nötiges Vorwissen in den Grundlagen geschaffen, um anschließend den Ist-Zustand zu analysieren. Aus dieser Analyse wurden die kritischen Punkte herausgezogen und in Anforderungen umformuliert. Diese Anforderungen wurden anschließend mit Hilfe eines Entwurfs umgesetzt. Anhand dieses Entwurfs wurde die Umstellung auf die neue Formulartechnologie erläutert.

Wichtige Anforderungen waren eine Vereinheitlichung der verschiedenen Varianten der Lieferantenerklärung, ohne Verluste seitens des Inhaltes. Anhand einer Analyse der Elemente des Smart Forms-Formulars konnten redundante Inhalte identifiziert und zusammengeführt beziehungsweise, wenn nötig, komplett entfernt werden. Gesellschaftsspezifische Inhalte wurden mit Hilfe von Anzeigebedingungen beibehalten.

Unabhängig vom Inhalt des Formulars, sollte die technische Umsetzung ebenfalls einheitlich und übersichtlich durchgeführt werden. Durch eine einheitliche Benennung der Elemente des \ac{PDF}s wurden der Inhalt und die Funktionsweise verdeutlicht. Ansonsten wurde darauf geachtet, dass gleiche Funktionen in einer festgelegten Weise umgesetzt wurden. Beispielsweise wurden alle statischen Texte in Textbausteine ausgelagert. Obendrein wurden inaktive Elemente des alten Formulars nicht in das neue übernommen, um so die Übernahme von Altlasten, welche eventuell nicht weiter verwendet werden, zu vermeiden. Im Anhang \ref{AN:lle-pdf-1}~-~\ref{AN:lle-pdf-4} ist ein Beispiel der \ac{LLE} mit Interactive Forms dargestellt.

Zusammenfassend ist die Harmonisierung des Dokumentes gelungen. Durch die vielseitigen Funktionen des Adobe \ac{PDF}s konnte ein Standard gebildet werden, welcher den Prozess der Lieferantenerklärung vereinheitlicht. Eine zeitliche Verkürzung von Änderungsprozesses ist zu erwarten, kann jedoch nicht gemessen werden. Hierfür muss es möglich sein die Bearbeitungszeiten verschiedener SAP-Tickets zu vergleichen, was durch die variablen Aufgabenstellungen und Anforderungen der Tickets nicht möglich ist.

Unabhängig von diesem Formular, gibt es noch viele weitere Dokumente im SAP-System von \ac{ABB}, welche immer noch mit Smart Forms auf verschiedene Weise umgesetzt sind. Auf Basis dieser Arbeit können nach und nach diese Dokumente mit den \ac{PDF}s von Adobe vereinheitlicht werden.

 
