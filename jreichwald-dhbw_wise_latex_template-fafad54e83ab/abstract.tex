\chapter*{Kurzfassung}
\begingroup
\begin{table}[h!]
\setlength\tabcolsep{0pt}
\begin{tabular}{p{3.7cm}p{11.7cm}}
Titel & \DerTitelDerArbeit \\
Verfasser/in: & \DerAutorDerArbeit \\
Kurs: & \DieKursbezeichnung \\
Ausbildungsstätte: & \DerNameDerFirma\\
\end{tabular}
\end{table}
\endgroup


In einem globalen Großkonzern wie der ABB ist es von großer Wichtigkeit Prozesse zu standardisieren. Gerade im Bereich der IT ist dies wichtig, da die Effektivität durch vereinheitlichte Abläufe über verschiedene Organisationseinheiten mit gleichbleibender Qualität gesteigert wird. Dadurch entstehen einheitliche Lösungswege, welche die Kosten reduzieren indem Einzelfallbearbeitungen entfallen.

Ziel dieser Arbeit ist eine Vereinheitlichung von mehreren gesellschaftsspezifischen Dokumenten zu einem standardisierten Formular.
Dabei wird darauf geachtet, dass die ältere Technologie der Dokumentenerstellung im SAP ERP - Smart Forms - durch die aktuellen Adobe Interactive Forms ersetzt wird. 

Zunächst muss der Ist-Zustand evaluiert werden, um die verschiedenen Varianten des Dokuments zusammenzuführen. Anschließend werden Anforderungen definiert und in einem Entwurf zusammengefasst. Die Umsetzung dieses Entwurfs beginnt mit der Erstellung der Schnittstelle zum Bereitstellen der benötigten Daten für das Formular. Mit Hilfe dieser Schnittstelle wird im Anschluss das Dokument inhaltlich aufgebaut und ein Layout konfiguriert.
Die Konfiguration beachtet die dynamische Inhaltsmenge des Dokuments, sowie spezifische Anforderungen der Gesellschaften wie beispielsweise Logos und Fußtexte. 
 
Das Umsetzen dieses standardisierten Formulars ist in dieser Arbeit geglückt. Es wird davon ausgegangen werden, dass dieses Formular in absehbarer Zeit von allen Gesellschaften gemeinsam eingesetzt wird. Die einheitliche Benennung von Elementen des Formulars, sowie ein strukturiertes Vorgehen bewirken eine Aufwandsminderung zukünftiger Änderungen. Eine Verbesserung der IT-Prozesse bezüglich dieses Formulars ist somit vorauszusehen.


