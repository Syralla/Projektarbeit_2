\chapter*{Kurzfassung}
\begingroup
\begin{table}[h!]
\setlength\tabcolsep{0pt}
\begin{tabular}{p{3.7cm}p{11.7cm}}
Titel & \DerTitelDerArbeit \\
Verfasser/in: & \DerAutorDerArbeit \\
Kurs: & \DieKursbezeichnung \\
Ausbildungsstätte: & \DerNameDerFirma\\
\end{tabular}
\end{table}
\endgroup

In einem Großkonzern wie der ABB ist es von großer Wichtigkeit Prozesse zu standardisieren. Gerade im Bereich der IT ist dies wichtig, da die Effektivität durch vereinheitlichte Abläufe über verschiedene Organisationseinheiten mit gleichbleibender Qualität gesteigert wird, Reduktion von Kosten erfolgt, da Einzelfallbearbeitung entfällt und Lösungswege einheitlich genutzt werden können und die Standardisierung zukünftige Prozessverbesserungen durch die Messbarkeit ermöglicht. Im Rahmen dieser Arbeit soll ein Dokument erstellt werden, welches automatisch aus dem \ac{ERP} System "`SAP"' heraus erstellt wird. Dieses PDF soll sich dynamisch an den erstellenden Bereich der ABB anpassen. Es werden zunächst die Grundlagen der PDF Erstellung im SAP erläutert und die benutzte Technologie wird erklärt. Anschließend wird der Ist-Zustand untersucht und kritisch beurteilt. Zudem wird ein Soll-Konzept erstellt werden und eine Vorgehensweise geplant. Anhand dieses Plans wird anschließend die Umsetzung eines neuen Formulars konzipiert und aufgestellt. Abschließend wird das entstandene PDF mit dem Soll-Konzept abgeglichen und ein Fazit wird gezogen.


