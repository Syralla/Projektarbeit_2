\chapter*{Kurzfassung}
\begingroup
\begin{table}[h!]
\setlength\tabcolsep{0pt}
\begin{tabular}{p{3.7cm}p{11.7cm}}
Titel & \DerTitelDerArbeit \\
Verfasser/in: & \DerAutorDerArbeit \\
Kurs: & \DieKursbezeichnung \\
Ausbildungsstätte: & \DerNameDerFirma\\
\end{tabular}
\end{table}
\endgroup

In einem Großkonzern wie der ABB ist es von großer Wichtigkeit Prozesse zu standardisieren. Gerade im Bereich der IT ist dies wichtig, da so das warten und Problemlösen optimiert wird. Im Rahmen dieser Arbeit soll ein Dokument erstellt werden, welches automatisch aus dem ERP System SAP heraus erstellt wird. Diese PDF soll dynamisch an den Erstellenden Bereich der ABB anpassen. Es werden zunächst die Grundlagen der PDF Erstellung im SAP erläutert und die benutzte Technologie wird erklärt. Anschließend wird der Ist-Zustand untersucht und kritisch beurteilt. Ein Soll-Konzept wird erstellt werden und eine Vorgehensweise geplant. Anhand dieses Plans wird anschließend das konzipiert und aufgestellt. Abschließend wird das entstandene PDF mit dem Soll-Konzept abgeglichen und ein Fazit wird gezogen.


